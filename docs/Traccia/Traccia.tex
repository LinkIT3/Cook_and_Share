\documentclass[a4paper]{article}
\raggedright
\begin{document}
\title{Social network per ricette}
\author{Matteo Carriera}
\date{}


\maketitle
Applicazione per la gestione di ricettario con le funzionalità di un social network. \newline

Le funzionalità includono:
\begin{itemize}
    \item Ricerca di una ricetta per nome
    \item Ricerca di una ricetta per ingredienti
    \item Ricerca di una ricetta per autore
    \item Ricerca di una ricetta per paese di origine
    \item Creazione di una ricetta
    \item Modifica delle ricette create
    \item Creazioni di varianti delle ricette
    \item Possibilità di seguire gli autori delle ricette
    \item Visualizzazione delle ricette degli autori seguiti o delle ultime ricette postate
    \item Creazione di un account
\end{itemize} 

\vspace{3mm}
L'applicazione può essere usata da utenti anonimi e registrati. \newline

Gli utenti anonimi possono: 
\begin{itemize}
    \item Visualizzare le ultime ricette postate
    \item Cercare ricette
\end{itemize}

\vspace{3mm}

Gli utenti registrati possono eseguire tutte le azioni degli utenti anonimi e in più possono:
\begin{itemize}
    \item Creare ricette
    \item Creare varianti delle ricette già esistenti
    \item Modificare le ricette create
    \item Seguire degli autori
    \item Visualizzare le ultime ricette postate dagli autori seguiti
    \item Personalizzare il proprio profilo con:
    \begin{itemize}
        \item Foto del profilo
        \item Banner del profilo
        \item Descrizione dell'utente
    \end{itemize}
\end{itemize}

\vspace{3mm}

L'utente amministratore può:
\begin{itemize}
    \item Creare nuovi utenti
    \item Eliminare gli utenti
    \item Dare e/o rimuovere il titolo di critico gastronomico agli utenti
    \item Modificare le informazioni degli utenti
    \item Creare le ricette
    \item Modificare le ricette
    \item Eliminare le ricette
\end{itemize}
\end{document}